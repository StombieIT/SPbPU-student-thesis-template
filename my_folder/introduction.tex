\chapter*{Введение} % * не проставляет номер
\addcontentsline{toc}{chapter}{Введение} % вносим в содержание

	В последние годы видеоконтент стал одним из основных источников потребления информации людьми. Технологии видеостриминга используются в различных сферах, включая бизнес, образование, медицину и индустрию развлечений, и играют в них большую роль. Видеоконтент зачастую используется менеджерами по продукции различных компаний для повышения внутренних ключевых показателей эффективности (KPI), включая важные продуктовые метрики: вовлечённость пользователей, удержание пользователей, брендовое восприятие. Всё это говорит о том, что видео является важным инструментом повышения конверсии восприятия людьми контента.

	Решаемые при обработке и воспроизведении видео, являются достаточно нетривиальными из-за возникающей высокой сетевой и вычислительной нагрузки. В связи с чем для многих организаций в различных отраслях становится нерентабельным создание и поддержка собственных систем хостинга и стриминга видео и возникает потребность в использовании уже имеющихся решений.

	Однако большинство современных сервисов видеостриминга не поддерживает воспроизведение нескольких синхронизированных потоков видео в одном экземпляре видеоплеера, хотя такая потребность возрастает с каждым годом, так как производимый пользователями контент становится технологически сложнее.
	
	\textbf{Актуальность}
	
	На текущий момент проблема синхронизации нескольких видеопотоков решается со стороны бизнеса-потребителя сервисов видеохостинга с помощью средств монтажа. Функциональная возможность переключения между несколькими синхронизированными видеопотоками позволит пользователю самостоятельно выбирать предпочтительные ему фрагменты контента, что положительно скажется на продуктовых и непродуктовых метриках потребления контента в различных сферах: медицина - улучшенный анализ сложных процедур и операций, образование - повышение эффективности обучения благодаря возможности сопоставления видеоматериала и презентации, субтитров, бизнес и сфера развлечений - повышение времени, проведённого пользователем (time spent).
	
	\textbf{Цель исследования:}
	
	Создание приложения с системой адаптивного стриминга видео в нескольких синхронизированных потоках с возможностью навигации между ними, распределённой сегментированной загрузкой контента и редактором синхронизации видеопотоков.

	\textbf{Задачи}

	\begin{enumerate}[1.]
		\item Разработка архитектуры системы распределённой сегментированной загрузки и раздачи видеоконтента;
		\item Разработка головного сервиса для обработки пользовательских запросов, инициации сессий загрузки, распределения асинхронных задач между другими сервисами, формирования пользовательской выдачи и обновления в реальном времени статуса загрузки для клиента;
		\item Реализация сервиса для распределённой обработки загрузки видеоконтента по сегментам и преобразования полученного видеоконтента в нужный кодек и в форматы различных протоколов адаптивного стриминга (DASH, HLS) для поддержания кроссплатформенности клиентов;
		\item Разработка сервиса для хранения и раздачи видеоконтента;
		\item Разработка вспомогательных сервисов для очистки прерванных сессий загрузки и удаления неиспользуемого контента (шедулеров);
		\item Разработка архитектуры клиентских приложений для адаптивного стриминга видео и редактора синхронизации видеопотоков;
		\item Разработка видеоплеера с поддержкой различных потоковых форматов в разных браузерах и синхронизации между видеопотоками;
		\item Разработка браузерного редактора синхронизации видеопотоков;
		\item Разработка и настройка среды контейнеризации для автоматизированного развёртывания серверных и клиентских сервисов, СУБД, брокеров сообщений и балансировщиков нагрузки;
		\item Проведение нагрузочного и end-to-end тестирования и апробации реализованного приложения;
	\end{enumerate}