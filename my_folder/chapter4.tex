\chapter{Заключение}

	В данной работе был проведён анализ предметной области, в результате которого были определены технологии, характерные для данной области, и подходы, которые легли в основу формирования требований и реализации системы.

	На этапе проектирования были сформулированы функциональные и нефункциональные требования к системе, выполнена декомпозиция системы на независимые сервисы и описаны принципы их взаимодействия, предложена архитектура приложения, обеспечивающая масштабируемость, отказоустойчивость и безопасность, обоснован выбор технологий и инструментов для реализации каждого компонента, включая брокер сообщений, систему управления базами данных, средства кэширования и балансировки нагрузки, разработаны схемы взаимодействия между сервисами.

	На этапе разработки реализованы ключевые компоненты серверной части, включая API-шлюз, сервисы загрузки, хранения и раздачи, и транскодирования видеоконтента, обеспечена поддержка асинхронной обработки видеосегментов с применением очередей сообщений, реализован планировщик задач для автоматической очистки устаревших сессий и видео, созданы клиентские приложения: видеоплеер, обеспечивающий воспроизведение нескольких синхронизированных потоков, и редактор синхронизации, реализующий создание, редактирование и загрузку видеопотоков.

	Таким образом, в результате выполнения данной работы была спроектирована и реализована система распределённой сегментированной загрузки и воспроизведения видеоконтента в нескольких синхронизированных потоках. Результаты данной работы планируется использовать для последующего тестирования и апробации системы.
